local search might be worth trying empirically 

> -- try to find a (any) node, q, of the quantum circuit
>     graph which is currently mapped to a node, x,
>     of the network topology graph,
>     such that there is neighbor node, y, of the node x
>     in the network topology
>     graph such that if you map q to y intead of x, and
>     keep the rest of the mapping the same, then the total
>     cost of the mapping strictly decreases
>     (where the cost is measured as the sum of the distances, in
>      the network topology graph, of
>      the two nodes to which the map sends each edge of the
>      quantum circuit graph).
>
> -- If no such "improvable" node q exists, and hence
>     the local improvement iterations have stopped,
>     then output the current mapping as a "local optimum".


 no guarantee that this local optimum will be anywhere
> close to the global opitimum, but it is worth trying this general "local
> search" strategy as a heuristic, to see if it results in reasonable
> mappings on typical cases


if you are actually going to implement it, then I would suggest that in
the first phase you could try doing something a little bit more clever
than just mapping all nodes of the quantum circuit graph uninformly at
random to the nodes of the network topology.

Instead, you could, for example, map all the nodes associated with
a single q-bit (i.e., one horizontal line) to the same node of
the network topology (a node which could be randomly chosen), etc.
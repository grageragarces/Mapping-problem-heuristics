circuit graph = hypergraph generated from a quantum circuit with the hdh library
network graph = graph that showcases the network stuctured of interconnected quantum chips

network graph has node "self loops"
circuit graph is directed

each node in the network graph has a capacity that cannot be exceeded

objective: map circuit graph to network graph such that:
 - circuit nodes are mapped to network nodes without exceeding capacity per network node 
 - no two circuit nodes that are interconnected can be mapped to network nodes that are not interconnected
    - two network nodes are considered interconnected even if the don't share a direct edge, but rather a path
- minimize the cost of this mapping: cost = number of edges between mapped nodes in the network graph
 - if two nodes are mapped through a path their cost is equivalent to the sum of edges separating them in the network graph

 It is our belief that this problem is at least NP-hard:

 > > Hi Maria,
> >
> > This email is just to confirm what I said I suspected in our meeting today:
> > the specific problem we discussed toward the end of our meeting,
> > which is a very special case of your general problem, is already NP-hard.
> >
> > Namely, consider just a 2-node network topology, with an edge between the two
> > nodes (and "self-loops" at each of the two nodes), where the "capacity" of
> > each of the two nodes is exactly (n/2), where n denotes the total number of
> > nodes of the quantum circuit graph.
> >
> > Our goal is to map the n-nodes of the quantum circuit
> > to the 2 nodes of the network topology, in such a way that
> > exactly half the nodes of the quantum circuit are mapped to each
> > of the two nodes of the network topology,  and we want to do so in such a way
> > as to *minimizes* the number of edges whose two
> > endpoint  are mapped to two distinct nodes in the network topology
> > (in other words, we want to minimize the size of the "cut" of
> > our quantum circuit graph, while splitting its set of nodes evenly
> > into two equal-sized sets).
> >
> > The above problem is already NP-hard.
> >
> > Indeed, if you look at the details of the proof of the sparsest (or,
> > equivalently via a simple transformation, the densest) cut problem given in
> > the following paper (Titled "The complexity status of problems related to
> > sparsest cut"; See section 2 of the paper), then that proof already shows
> > that the unit-edge-capacity sparsest cut problem is NP-hard (via a reduction
> > from the maximum-cut problem which is well known to be NP-hard), already in
> > the special case where the only possible sparsest cuts must split the set of
> > nodes in the graph into two exactly equal-sized sets.
> >
> > Here's the link for the paper:
> >
> >https://eur02.safelinks.protection.outlook.com/?url=https%3A%2F%2Flink.springer.com%2Fchapter%2F10.1007%2F978-3-642-19222-7_14&data=05%7C02%7C%7C47f4365b88734a67fca208ddb3300a32%7C2e9f06b016694589878910a06934dc61%7C0%7C0%7C638863741084775691%7CUnknown%7CTWFpbGZsb3d8eyJFbXB0eU1hcGkiOnRydWUsIlYiOiIwLjAuMDAwMCIsIlAiOiJXaW4zMiIsIkFOIjoiTWFpbCIsIldUIjoyfQ%3D%3D%7C0%7C%7C%7C&sdata=573ZqTkmxaKZhufbgof%2FQ%2B6GUnBBHXlvstValvAvqEs%3D&reserved=0
> 089718ceb44aefac608ddb319921e%7C2e9f06b016694589878910a06934dc61%7C0%7C0%7C638863644583451882%7CUnknown%7CTWFpbGZsb3d8eyJFbXB0eU1hcGkiOnRydWUsIlYiO
> iIwLjAuMDAwMCIsIlAiOiJXaW4zMiIsIkFOIjoiTWFpbCIsIldUIjoyfQ%3D%3D%7C0%7C%7C%7C&sdata=2nSlyrSaacW8J3O4zXsenIVPldmXOSrDbzCSJnjblcE%3D&reserved=0
> >
> > Of course this still leaves open the question of what is the best
> > approximation we could hope for, in terms of approximating the minimum number
> > of edges that must cross the cut, while respecting the capacity constraint on
> > each side of the cut.
> >
> > I think it will not be too difficult to show that we cannot hope for a
> > polynomial time algorithm whose worst-case approximation ratio is better than
> > the existing O(sqrt(log n)) approximation ratio for the Sparsest Cut problem,
> > without also improving the current best approximation ratio on sparest cut,
> > which is a long-standing open problem. That's unfortunately a rather negative
> > result, because an approximation
> > ratio of O(\sqrt(log n)) isn't good in general (and, as I mentioned
> > before, the Arora-Rao-Vazirani algorithm that achieves that approximation
> > ratio is really terribly slow, despite being theoretically in polynomial
> > time).
> >
> > And all this is for an extremely special case of your more general problem,
> > namely the case where the network topology is just a 2-node
> > graph.  So, it doesn't bode well for finding very good algorithms
> > for the general case.
> >